\documentclass[12pt, twosided]{report}  % type of Documents


%_____________________________________PACKAGES__________________________________
\usepackage[utf8]{inputenc} 	% input encoding [utf8]
\usepackage[english]{babel} 	% setting spell-check for English
\usepackage{geometry} 			% for page margins
\usepackage{fancyhdr} 			% for header and footer
\usepackage{url} 				% package to include urls
\usepackage{datetime} 			% For inserting date and time
\usepackage{graphicx} 			% For inserting graphics
\usepackage{float} 				% for accurate placement of figures and tables
\usepackage[labelfont=bf,textfont=bf]{caption} % for having the captions in bold
\usepackage{amsmath} %for equations
\usepackage[hidelinks]{hyperref} % for adding hyperlinks (withouht ugly boxes)
\usepackage{nomencl} 			% For Nomenclature 
\makenomenclature
\usepackage{amssymb} 			% symbols for nomenclature
\usepackage{etoolbox} 			% For Creating Categories in Nomenclature
\usepackage{caption} 								% For subfigures
\usepackage{subcaption} 							% For subfigures
\usepackage{tcolorbox}								% For boxed text
\usepackage{soul}

%_______________________________________________________________________________


% PAGE GEOMETRY
\geometry{left=1cm,right=1cm,top=1cm,bottom=1cm,includeheadfoot}

% Miscellanious
\setcounter{secnumdepth}{5} % To have numbering in the subsubsection and beyond


%_____________________________HEADER AND FOOTER_______________________
\fancypagestyle{mypagestyle}{
\fancyhead{}
\fancyfoot{}
\fancyhead[L]{\rightmark}
\fancyhead[R]{Machine Learning Project on EHR Data}
\fancyfoot[R]{\today}
\fancyfoot[C]{\thepage}
\fancyfoot[L]{Divyansh Chahar}
\renewcommand{\headrulewidth}{0pt}
\renewcommand{\footrulewidth}{0.8pt}
}
%_____________________________________________________________________


%% This code creates the groups
% -----------------------------------------
\renewcommand\nomgroup[1]{%
	\ifstrequal{#1}{A}{\item[\Large\bfseries{Greek Characters}]}{%
	\ifstrequal{#1}{B}{\vspace{10pt} \item[\Large\bfseries{Roman Characters}]}{%
	\ifstrequal{#1}{C}{\vspace{10pt} \item[\Large\bfseries{Acronyms}]}{}}}%
}
% -----------------------------------------


\pagestyle{mypagestyle} % For Header and Footer
\renewcommand\thesection{\arabic{section}} % Section Numbers Stating From 1
\renewcommand{\thefigure}{\thesection-\arabic{figure}} % to include section numbers in figures

%%%%%%%%%%%%%%%%%%%%%
% TITLE PAGE BEGINS %
%%%%%%%%%%%%%%%%%%%%%

\begin{document}
	\begin{titlepage}
		\newgeometry{a4paper,top=3cm,bottom=1cm,right=3cm,left=3cm} % Setting Page Dimensions
		\begin{center}
			{\LARGE \textbf{Machine Learning Project using EHR Data}} \\
			
			\hrulefill
		
			\textbf{Capstone Project}
			\\
			\textbf{Data Science Batch of 8-Feburary 2020} 
			
			\null
			
			Divyansh Chahar
			
			\vfill
			
			\href{https://www.linkedin.com/in/divyanshchahar/}{\includegraphics[width=0.25\linewidth]{./images/icons/my_qrcode.eps}}
		
			\null
		
			\href{https://www.linkedin.com/in/divyanshchahar/}{\includegraphics[width=0.025\linewidth]{./images/icons/linkedin_logo.eps}}
			\href{https://www.linkedin.com/in/divyanshchahar/}{https://www.linkedin.com/in/divyanshchahar/}
			
			\null
			
			\href{https://www.linkedin.com/in/divyanshchahar/}{\includegraphics[width=0.025\linewidth]{./images/icons/github_mark.eps}}
			\href{https://github.com/divyanshchahar}{https://github.com/divyanshchahar}
			
			\vfill
			
			\today
			
		\end{center}
	\end{titlepage}

%%%%%%%%%%%%%%%%%%%
% TITLE PAGE ENDS %
%%%%%%%%%%%%%%%%%%%

\restoregeometry

%\nomenclature[A]{$N$}{Total Population}
%\nomenclature[A]{$t$}{Time in days}
%\nomenclature[A]{$S_t$}{Individual Susceptible at time $t$}
%\nomenclature[A]{$I_t$}{Individual Infected at time $t$}
%\nomenclature[A]{$R_t$}{Individual Removed at time $t$}
%\nomenclature[A]{$\overline{I}$}{Average Infected}
%\nomenclature[A]{$R$}{Reproduction Number}
%\nomenclature[A]{$A_n$}{Moving Average}
%\nomenclature[A]{x}{Parameter Under Consideration}
%\nomenclature[A]{$G_t$}{Growth Rate at time $t$}
%\nomenclature[B]{$\beta$}{Diseaese Transmission Rate Constant}
%\nomenclature[B]{$\gamma$}{Recovery Rate}
%
%\printnomenclature

\pagebreak

\section{Introduction}
Introduction goes here.

\section{EDA approach}
EDA for this model was divided into 2 main parts i.e. understanding the entries in a particular file and understanding the insights.
 
\subsection{Understanding file entries}

\subsubsection{allergies.csv}


\begin{table}[H]
	\begin{tabular}{p{4cm}|p{15cm}}
		\textbf{File}           & \textbf{Operation} \\ \hline
		allergies\_uvc1.csv     & Unique Values Count on "DESCRIPTION"   \\
		allergies\_uvc2.csv     & Unique Values Count on "PATIENT"  \\
		allergies\_nuvc1.csv    & Unique Values Count on "PATIENT" on dataframe filtered by null   values in "STOP"   \\
		allergies\_unwanted.csv & Count of unwanted data in the dataframe  \\
	\end{tabular}
	\caption{Files with results of EDA on allergies.csv}
\end{table}

\begin{itemize}
	\item \textbf{Allergy to mould} is the most frequent entry and \textbf{Allergy to soya} is the least common entry (see \textit{allergies\_uvc1.csv})
	
	\item It is possible for a single patient id to be listed against multiple enteries in \textit{allergies.csv}, (\textit{see allergies\_uvc2.csv}) however is it possible for a patient to have multiple allergies remain to be seen.
	
	\item It is possible for a patient to have multiple ongoing allergies (see \textit{allergies\_nuvc1})
	
	\item "STOP" column in \textit{allergies.csv} have some null values, but this cannot be attributed to unwanted or missing data, it represents entries with ongoing allergies(see \textit{allergies\_unwanted.csv}).
	
\end{itemize}

\subsubsection{careplans.csv}

\begin{table}[H]
	\centering
	\begin{tabular}{p{4.5cm}|p{15cm}}
		\textbf{File}                & \textbf{Operation} \\ \hline
		careplans\_uvc1.csv & Unique Values Count on "PATIENT"   \\
		careplans\_uvc2.csv & Unique Values Count on "DESCRIPTION"  \\
		careplans\_uvc3.csv    & Unique Values Count on "REASONDESCRIPTION"   \\
		careplans\_nuvc1.csv    & Unique Values Count on "PATIENT" on dataframe filtered by null values in "STOP" \\
		careplans\_nuvc2.csv    & Unique Values Count on "DESCRIPTION" on dataframe filtered by null values in "REASONDESCRIPTION"  \\
		careplans\_unwanted.csv & Count of unwanted data in the dataframe  \\
	\end{tabular}
	\caption{Files with results of EDA on careplans.csv}
\end{table}


\begin{itemize}
	\item It is possible for a patient to  have multiple careplans (see \textit{careplans\_uvc1.csv}), however further investigation is required to check if it is possible for a patient to have multiple ongoing careplans.
	
	\item It is possible for a single patient to have multiple ongoing careplans (see \textit{careplans\_nuvc1.csv})
	
	\item To study the most common reasons careplans are used for, it was deterimined to perform a unique Values count on the columns named "DESCRIPTION" and "REASONDESCRIPTION" of the \textit{careplans.csv}. Perfomring a unique values count on either one of the column will not be sufficient as \hl{single value in the "DESCRIPTION" colummn could be listed  against multiple entries in the "REASONDESCRIPTION" column.}
	
	\item Careplans are most commonly used for \textbf{respiratory therapy} (see \textit{careplans\_uvc2.csv})
	
	\item Careplans are most commonly used for treatment of \textbf{Acute Bronchitis(Disorder)} (see \textit{careplans\_uvc3.csv}) which is a respiratory disorder thus justifying the fact that Respiratory Therepy is the most commonly occuring value in the "DESCRIPTION" column
	
	\item The "STOP" column contains NaN values but it could be attributed to ongoing careplans and cannot be considered as unwanted or bad data. The "REASONDECSRIPTION" column also contains NaN values it represents the use of careplans for non-medical reasons. \hl{Thus we can say that \textit{careplans.csv} does not contains any unwanted data} (see \textit{careplans\_unwanted.csv}).
	
	\item For non-medical reason (i.e. when "REASONDESCRIPTION" has a NaN value) the careplans are most commonly used for \textbf{Self-care interventions (procedure)} (see \textit{careplans\_nuvc2.csv}).
	
\end{itemize}

\subsubsection{conditions.csv}

\begin{table}[H]
	\centering
	\begin{tabular}{p{5cm}|p{14.5cm}}
		\textbf{File}                & \textbf{Operation} \\ \hline
		conditions\_uvc1.csv     & Unique Values Count on "DESCRIPTION"   \\
		conditions\_nuvc1.csv    & Unique Values Count on "DESCRIPTION"  on dataframe filtered by null values in "STOP" \\
		conditions\_psudo\_chronic.csv & List of psudo chronic conditions i.e conditions which have no "STOP" values in some cases but do have a "STOP" value somewhere else.
	\end{tabular}
	\caption{Files with results of EDA on conditions.csv}
\end{table}

\begin{itemize}
	\item Viral Sinusitis (disorder) was the most common condition (see \textit{conditions\_uvc1.csv})
	
	\item It was observed that the "STOP" column in the \textit{careplans.csv} had some null values, thus giving the impression that certain conditions could be chronic in nature thus the dataset was filtered by null values in the "STOP" column and unique values count was performed on the "DESCRIPTION" column, the results of this operation are stored in \textit{condition\_nuvc1.csv}. However this approach has a major drawback, \hl{some conditions which do not have a stop date in certain cases do have stop dates in some other cases.} 
	
	\item In order to check weather a condition is trully chronic in nature or not, further investigation was required. The dataset was checked to see if any of the conditions in the \textit{conditions\_nuvc1.csv} has stop dates anywhere in the dataset. The enteries pertaining to this investigation are recoded in \textit{conditions\_psudo\_chronic.csv}.
	
	\item To get the list of final chronic conditions, a difference operation was performed between the list of conditions in \textit{conditions\_STOP\_null.csv} and \textit{conditions\_nuvc1.csv}. The resultant conditions were recorded in \textit{conditions\_chronic.csv} stored in the \textit{preprocessed} folder
\end{itemize}

\subsubsection{encounters}

\begin{table}[H]
	\centering
	\begin{tabular}{p{5cm}|p{14.5cm}}
		\textbf{File}             & \textbf{Operation} \\ \hline
		encounters\_uvc1.csv      & Unique Values Count on "REASONDESCRIPTION"   \\
		encounters\_uvc2.csv      & Unique Values Count on "DESCRIPTION"   \\
		encounters\_uvc3.csv      & Unique Values Count on "ENCOUNTERCLASS"   \\
		encounters\_nuvc1.csv     & Unique Values Count on "DESCRIPTION" filtered by NaN values "REASONDESCRIPTION" \\
		encounters\_nuvc2.csv     & Unique Values Count on "ENCOUNTERCLASS" filtered by NaN values "REASONDESCRIPTION" \\
		encounters\_ct1.csv       & Cross Tabulation operation between "REASONDESCRIPTION" and "ENCOUNTERCLASS" \\
		encounters\_ct2.csv       & Cross Tabulation operation between "DESCRIPTION" and "ENCOUNTERCLASS" \\
		encounters\_ct3.csv       & Cross Tabulation between "DESCRIPTION" and "REASONDESCRIPTION" column \\
		encounters\_unwanted. csv & Count of Unwanted data on the dataframe  
	\end{tabular}
	\caption{Files with results of EDA on encounters.csv}
\end{table}

\begin{itemize}
	\item Most encounters happen because of non-medical reasons i.e. \textbf{NaN values} and the second most common cause of encounters is \textbf{Normal Pregenancy} (see \textit{encounters\_uvc1.csv}).
	
	\item As discussd above the most encounters are due to non-medical reasons, a further analysis reveals that most encounters can be classified as \textbf{Well child visit (procedure)} (see \textit{encounters\_uvc2.csv}) .
	
	\item For most of the encounters the patient was in an ambulatory state i.e. the patient is not bed ridden and can walk (see \textit{encounters\_uvc3.csv}).
	
	
	\item \textit{encounters.csv} was also analyzed for unwanted data and the output was stored in \textit{encounters\_unwanted.csv}. It was observed that only "REASONDESCRIPTION" had \textbf{NaN values} and "PAYER\_COVERAGE" had some zeros. \hl{It is possible that a patient had no insurance coverage at all hence the zeros in the "PAYER\_COVERAGE" column cannot be treated as unwanted values values however the null values in "REASONDESCRIPTION"  column need to be investigated further.} Thus at this point it can't be stated with absolute certainty that \textit{encounters.csv} does not have any unwanted data (see \textit{encounters\_unwanted.csv}). 
	
	\item For encounters which are not because any major medical purpouse the most common enounters are for  \textbf{Well child visit (procedure)} (see \textit{encounters\_nuvc2.csv}) . After comparing the data in \textit{encounters\_uvc2.csv} and \textit{encounters\_nuvc1.csv} we can say that all the encounters of the type \textbf{Well child visit (procedure)} are for non-medical reasons.
	
	\item It was also observed that most of the encounters for non-medical reason are most commonly of the type \textbf{wellness} (see \textit{encounters\_nuvc2.csv})
	
	\item On a closer observation it was noticed that a \hl{single entry in "DESCRIPTION" column could be listed against multiple entries in "REASONDESCRIPTION" and "ENCOUNTERCLASS".} Thus several cross tabulation operations were performed to understand the distribution of the data across multiple categories.
	
	\item Most of the entries in "REASONDESCRIPTION" and "DESCRIPTION" could be listed across multiple entries of "ENCOULNTERCLASS" (see \textit{encounters\_ct1.csv} and \textit{encounters\_ct2.csv}).
	
	\item It must also be noted that several entries in the "DESCRIPTION" column could be listed against more than one entry in the "REASONDESCRIPTION" column (see \textit{encounters\_ct3.csv}) .
	
	\item The "ENCOUTERCLASS" column has several unique values, we are particularly interested in the encounters where "ENCOUTERCLASS" is \textbf{emergency} and \textbf{urgentcare}.

\end{itemize}


\subsubsection{imaging\_studies.csv}

\begin{table}[H]
	\centering
	\begin{tabular}{p{5.5cm}|p{14cm}}
		\textbf{File}                  & \textbf{Operation} \\ \hline
		imaging\_studies\_uvc1.csv     & Unique Values Count on "BODYSTIE\_DECSRIPTION" \\
		imaging\_studies\_uvc2.csv     & Unique Values Count on "MODALITY\_DESCRIPTION" \\
		imaging\_studies\_ct1.csv      & Cross Tabulation operation on "BODYSITE\_DESCRIPTION" and "SOP\_DESCRIPTION" \\
		imaging\_studies\_ct2.csv      & Cross Tabulation operation on "MODALITY\_DESCRIPTION" and "SOP\_DESCRIPTION" \\
		imaging\_studies\_unwanted.csv & Count of Unwanted Data in Dataframe
	\end{tabular}
	\caption{Files with results of EDA on imaging\_studies.csv}
\end{table}


\begin{itemize}
	
	\item Thoracic Structutre (Body Study) is the most examined body structure (see \textit{imaging\_studies\_uvc1.csv}) .
	
	\item \textbf{Digital X-ray} is the most commonly performed procedure (see \textit{imaging\_studies\_uvc2.csv}).
	
	\item It can be noted that multiple tests could be performed on a single body part.

	\item \textit{Thoracic structure (body structure)} is the only value in "BODYSITE\_DECSRIPTION" that is listed against multiple values in the "SOP\_DESCRIPTION" column (see \textit{imaging\_studies\_ct1.csv}) . It must also be noted that "BODYSITE\_DESCRIPTION" also have values \textbf{Thoracic structure} and \textbf{thoracic} which are listed against a single value in "SOP\_DESCRIPTION"  column. \hl{However if these represent different body site or are just a different values for the the same body site is a topic of further analysis.} All ohter values in "BODYSITE\_DESCRIPTION" column are listed against a single value in "SOP\_DESCRIPTION" column. 
	
	\item \textit{imaging\_studies.csv} does not contains any unwanted data (see \textit{imaging\_studies\_unwanted.csv})
	
\end{itemize}

\subsubsection{immunizations.csv}

\begin{table}[H]
	\centering
	\begin{tabular}{p{5cm}|p{14.5cm}}
		\textbf{File}              & \textbf{Operation} \\ \hline
		immunization\_uvc1.csv     & Unique Values Count on "DESCRIPTION" column\\
		immunization\_unwanted.csv & Count of Unwanted Data on Dataframe
	\end{tabular}
	\caption{Files with results of EDA on immunizations.csv}
\end{table}


\begin{itemize}
	
	\item \textbf{Influenza  seasonal  injectable  preservative free} was the most frequently occurring entry (see \textit{immunization\_uvc1.csv}).

	\item \textit{immunization.csv} contains no unwanted data (see \textit{immunization\_unwanted.csv})
	
\end{itemize}

\subsubsection{medications.csv}


\begin{table}[H]
	\centering
	\begin{tabular}{p{4.5cm}|p{15cm}}
		\textbf{File} & \textbf{Operation} \\ \hline
		medications\_uvc1.csv     & Unique Values Count on "DESCRIPTION" column \\
		medications\_uvc2.csv     & Unique Values Count on "REASONDESCRIPTION" column \\
		medications\_ct1.csv      & Cross Tabulation operation on "DESCRIPTION" and "REASONDESCRIPTION" \\
		medications\_nuvc1.csv    & Unique Values Count on "DESCRIPTION" on dataframe filtered by NaN values in "REASONDESCRIPTION" \\
		medications\_unwanted.csv & Count of Unwanted data in the dataframe 
	\end{tabular}
	\caption{Files with results of EDA on medications.csv}
\end{table}


\begin{itemize}
	
	\item \textbf{Hydrochlorothiazide 25 MG Oral Tablet} is the most common medication (see \textit{medication\_uvc1.csv}).
	
	\item Most medications are prescribed for minor medical reasons i.e. "REASONDESCRIPTION" having a \textbf{NaN value}, the second most common reason is \textbf{Hypertension} (see \textit{medication\_uvc2.csv}).

	\item One medication could be used to treat different conditions (see \textit{medications\_ct1.csv})
	
	\item In the absence on of any major medical reason i.e. "REASONDESCRIPTION" column has \textbf{NaN value} the most frequent entry in the data base is \textbf{Nitroglycerin 0.4 MG/ACTUAT Mucosal Spray} (see \textit{medication\_nuvc1.csv})
	
	\item It was observed that "REASONCODE" contains some \textbf{NaN values}, some values in the "PAYER COVERAGE" and "TOTAL COST" columns are also \textbf{0}, although it is possible for a patient to have no health coverage, the possibility of "TOTAL COST" column having \textbf{0} might point towards the presence of bad data (see \textit{medications\_unwanted.csv}). Thus a further investigation is required to determine if ther is any bad data in the \textit{medication.csv}.
	
\end{itemize} 

\subsubsection{observations.csv}

\begin{table}[H]
	\centering
	\begin{tabular}{p{5cm}|p{14.5cm}}
		\textbf{File} & \textbf{Operations} \\ \hline
		observations\_uvc1.csv     & Unique Values Count on "DESCRIPTION" column \\
		observations\_unwanted.csv & Count of Unwanted Data in the Dataframe
		\end{tabular}
	\caption{Files with results of EDA on observations.csv}
\end{table}


\begin{itemize}

\item \textbf{Pain severity - 0-10 verbal numeric rating [Score] - Reported} is the most commonly performed procedure.

\item Based on the nature of the data it is difficult to draw any meaningful conclusion, thus a further analysis is required to draw meaningful insights.

\item Based on an initial analysis it was observed that "ENCOUNTER" and "UNITS" column has some \textbf{NaN values} and some values in the "VALUE" column are also zero. \hl{Thus based on an initial analysis it can be stated that \textit{observations.csv} does have some unwanted data.}

\end{itemize}

\subsubsection{procedures.csv}

\begin{table}[H]
	\centering
	\begin{tabular}{p{4.5cm}|p{15cm}}
		\textbf{File}            & \textbf{Operation} \\ \hline
		procedures\_uvc1.csv     & Unique Values Count on "DESCRIPTION" column \\
		procedures\_uvc2.csv     & Unique Values Count on "REASONDESCRIPTION" column \\
		procedures\_ct1.csv      & Cross Tabulation operation on "DESCRIPTION" and "REASONDESCRIPTION" \\
		procedures\_nuvc11.csv      & Unique Values Count on "DESCRIPTION" on dataframe filtered by NaN Values in "REASONDESCRIPTION" \\
		procedures\_unwanted.csv & Count of unwanted data on dataframe
	\end{tabular}
	\caption{Files with results of EDA on procedures.csv}
\end{table}

\begin{itemize}

	\item \textbf{Medication Reconciliation (procedure)} is the most frequently performed procedure (see \textit{procedures\_uvc1.csv}).
	
	\item Most of the procedures are performed due to \textbf{Normal Pregnenecy} (see \textit{procedures\_uvc2.csv})
	
	\item It was observed that "REASONDESCRIPTION" does contain some \textbf{NaN Values} (see \textit{procedures\_nuvc1.csv}). However these cannot be regarded as missing values as this represents the cases where there is no major medical reason.
	
	\item Even in the absence of any major medical reason it was observed that \textbf{Medication Reconciliation (procedure)} was most frequently performed procedure(see \textit{procedures\_nuvc1.csv}) .  
	
	\item On a closer obervation it could be observed that a procedure could be performed for multiple medical reasons (see \textit{procedures\_ct1.csv})
	
\end{itemize} 

%\bibliographystyle{ieeetr}
%\bibliography{myrefrences}

\end{document}